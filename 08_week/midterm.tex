\documentclass[11pt]{article}
\usepackage{amsmath, amssymb, graphicx, hyperref}
\usepackage{enumitem}
\setlist{nosep}
\usepackage[margin=1in]{geometry}

\title{Take-Home Midterm: Visualizing War, Regime Type, Distance, and Trade (R + GitHub)}
\author{}
\date{}

\begin{document}
\maketitle

\noindent \textbf{Overview.} This take-home midterm asks you to construct a small analytical dataset using the Correlates of War framework and visualize relationships between war, regime type, distance, and trade. Your goal is not to estimate causal effects, but to use visualization to explore patterns, diagnose structure, and communicate tradeoffs clearly.

\medskip
\noindent \textbf{Allowed resources.} You may use course materials, textbooks, documentation, and online resources.

\medskip
\noindent \textbf{AI restriction (required).} 

\begin{itemize}
\item You \textbf{may not} use AI tools (e.g., ChatGPT, Copilot, or similar systems) to write your answers. Your written interpretations and explanations must be your own. The exam is open book, note, web, AI. \textbf{However, you are responsible for understanding and explaining your code and results. However, you can use any resource to help you.}
\item You may not work with anyone else on the exam. This has to be your own work. 
\end{itemize} 

\medskip
\noindent \textbf{Provided dataset (required).} You will begin from a provided dyad-year dataset (not from \texttt{peacesciencer}). The dataset is a dyad-year panel with one row per country-pair per year.

\medskip
\noindent \textbf{Mini codebook (variables you will use).} The dataset includes the following columns:
\begin{itemize}
  \item \texttt{year}: calendar year.
  \item \texttt{ccode1}, \texttt{ccode2}: COW country codes for state 1 and state 2.
  \item \texttt{iso3c1}, \texttt{iso3c2}: ISO-3 abbreviations for state 1 and state 2.
  \item \texttt{capdist}: capital-to-capital distance (km). (May contain missing values.)
  \item \texttt{trade}: dyadic trade volume (may be zero; may contain missing values).
  \item \texttt{cowinteronset}: indicator (0/1) for an inter-state war onset in that dyad-year (may contain missing values).
  \item \texttt{cowinterongoing}: indicator (0/1) for an inter-state war ongoing in that dyad-year (may contain missing values).
  \item \texttt{polity21}, \texttt{polity22}: regime category for each state in the dyad-year. Values include \texttt{Democracy}, \texttt{Autocracy}, and \texttt{Anocracy}.
    \item \texttt{initiator1}, \texttt{initiator2}: Country that started the conflict. 
\end{itemize}

\medskip
\noindent \textbf{Important note on aggregation.} You will need to construct at least one \textbf{system-year} (or year-level) dataset from this dyad-year data. For example, you may aggregate dyad-year war indicators into counts per year (total wars per year; wars between democracies per year), and you may aggregate dyad-year relationships into year-level summaries as needed for visualization.




\medskip
\noindent \textbf{What to submit (in your GitHub repo).}
\begin{itemize}
  \item A script file (or Quarto/Rmd) that runs top-to-bottom:
  \begin{itemize}
    \item \texttt{scripts/midterm.R} \textit{(preferred)} or \texttt{midterm.qmd} / \texttt{midterm.Rmd}
  \end{itemize}
  \item A short write-up:
  \begin{itemize}
    \item \texttt{outputs/writeup.md}
  \end{itemize}
  \item Figures saved to:
  \begin{itemize}
    \item \texttt{figures/}
  \end{itemize}
\end{itemize}

\medskip
\noindent \textbf{Reproducibility requirements.}
\begin{itemize}
  \item Use an \textbf{R Project}.
  \item Use a \textbf{sequential, hard-coded workflow} (no user-defined functions).
  \item Save figures using \texttt{ggsave()} (no screenshots).
  \item Organize your repo with \texttt{scripts/}, \texttt{outputs/}, and \texttt{figures/}.
\end{itemize}

\medskip
\noindent \textbf{Visualization requirements.}
\begin{itemize}
  \item You must produce at least \textbf{four} figures, each answering a \textbf{different} analytical question.
  \item At least one figure must be shown in \textbf{two versions}:
  \begin{itemize}
    \item raw scale, and
    \item transformed scale (e.g., log, rate, proportion, binning).
  \end{itemize}
  \item You must use \textbf{color} intentionally:
  \begin{itemize}
    \item color mappings must be intuitive for the task,
    \item color choices must be accessibility-conscious (colorblind-friendly),
    \item and you must justify your color choices in writing.
  \end{itemize}
\end{itemize}

\section*{Tasks}

\begin{enumerate}

  \item \textbf{Data construction transparency (required).}

  Start from the provided dyad-year dataset and construct the analysis datasets you need for the tasks below. At minimum, you must construct:

  \begin{itemize}
    \item a \textbf{year-level (system-year)} dataset that supports ``wars per year'' summaries, and
    \item a \textbf{dyad-year} dataset (possibly filtered) that supports distance--trade--war visualizations.
  \end{itemize}

  \noindent \textbf{Requirement:} In your write-up, briefly describe how you aggregated from dyads to years (what you counted/summed, and how you handled missing values).

  \medskip
  \noindent In your write-up, include a section titled \textbf{Data construction decisions} listing:
  \begin{itemize}
    \item how you created your year-level dataset from dyad-year data (aggregation rule),
    \item the unit(s) of analysis you created (system-year vs dyad-year),
    \item and what observations you filtered or dropped (and why).
  \end{itemize}

  \item \textbf{System-level war trends (required).}

  Using your system-year dataset, create a visualization showing:
  \begin{itemize}
    \item the total number of wars per year, and
    \item the number of wars between democracies per year.
  \end{itemize}

  \medskip
  \noindent \textbf{Requirement:} You must justify your design choice (counts vs rates, smoothing vs no smoothing, scale choices) in 3--5 sentences.

  \item \textbf{Regime type comparisons (required).}

  Create at least one visualization comparing war patterns across dyad regime types:
  \begin{itemize}
    \item democracy--democracy,
    \item mixed dyads,
    \item autocracy--autocracy.
  \end{itemize}

  \medskip
  \noindent \textbf{Requirement:} Regime type must be encoded clearly (color or faceting). If you use color, justify palette choice and accessibility.

  \item \textbf{Distance and war (required).}

  Using dyad-year data, visualize the relationship between:
  \begin{itemize}
    \item distance between states, and
    \item war occurrence or war frequency.
  \end{itemize}

  \medskip
  \noindent \textbf{Requirement:} You must explain how your visualization approach addresses the fact that war is a rare event.

  \item \textbf{Trade, distance, and conflict (required).}

  Create a visualization that jointly considers:
  \begin{itemize}
    \item trade volume,
    \item distance,
    \item and war occurrence.
  \end{itemize}

  \noindent You may choose the form (scatter, faceting, binned summaries, transformed axes), but you must justify why your approach is informative given the distribution of the data.

  \item \textbf{Raw vs transformed scale comparison (required).}

  Choose one of your figures and produce two versions:
  \begin{itemize}
    \item a raw-scale version, and
    \item a transformed-scale version.
  \end{itemize}

  \noindent In your write-up, explain:
  \begin{itemize}
    \item what becomes easier to see after transformation,
    \item what becomes harder to interpret,
    \item and which version you would show to (i) a policymaker and (ii) a methods audience (with justification).
  \end{itemize}

  \item \textbf{One ``bad'' visualization (required).}

  Produce one intentionally misleading or poorly designed visualization related to this midterm.

  \medskip
  \noindent In your write-up, include 3--5 bullet points explaining precisely why it is misleading (e.g., inappropriate scale, poor color encoding, missing context, distortion, or ambiguous labeling).

  \item \textbf{Interpretation and narrative (required).}

  In \texttt{outputs/writeup.md}, write a coherent narrative (approximately 800--1200 words) that:
  \begin{itemize}
    \item introduces what you attempted to learn from the data,
    \item walks through your figures in a logical order,
    \item interprets what the figures show,
    \item and highlights limitations and design tradeoffs.
  \end{itemize}

  \noindent \textbf{Color requirement (repeat):} For each figure that uses color, include 1--2 sentences justifying:
  \begin{itemize}
    \item what color encodes,
    \item why color is appropriate,
    \item and why your palette is accessibility-conscious.
  \end{itemize}

  \item \textbf{GitHub submission (required; proof required).}

  Commit and push your work to GitHub.

  \medskip
  \noindent In your write-up, paste:
  \begin{itemize}
    \item the output of \texttt{git status} after committing (clean working tree), and
    \item the output of \texttt{git log -3} showing at least three commits.
  \end{itemize}

\end{enumerate}

\section*{Checklist (before submitting)}
\begin{itemize}
  \item Your code runs top-to-bottom in a fresh R session
  \item \texttt{scripts/midterm.R} (or \texttt{midterm.qmd}/\texttt{midterm.Rmd}) exists
  \item \texttt{outputs/writeup.md} exists and contains all required sections
  \item At least four distinct figures saved in \texttt{figures/}
  \item One figure appears in raw and transformed versions
  \item One intentionally ``bad'' visualization included with critique bullets
  \item GitHub proof included: \texttt{git status} (clean) and \texttt{git log -3}
\end{itemize}

\end{document}
