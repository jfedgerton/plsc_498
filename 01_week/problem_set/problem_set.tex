\documentclass[11pt]{article}
\usepackage{amsmath, amssymb, graphicx, hyperref}
\usepackage{enumitem}
\setlist{nosep}
\usepackage[margin=1in]{geometry}

\title{In-Class Problem Set: Environment Setup + Git/GitHub Workflow (R)}
\author{}
\date{}

\begin{document}
\maketitle

\noindent \textbf{Goal.} Set up a reproducible workflow on your machine so you can run course code, save outputs, and submit work through GitHub.

\medskip
\noindent \textbf{What to submit (in your GitHub repo).}
\begin{itemize}
  \item A short write-up: \texttt{outputs/setup\_check.md}
  \item A screenshot file: \texttt{outputs/terminal\_proof.png} (or \texttt{.jpg})
  \item A text file created by R: \texttt{outputs/r\_check.txt}
\end{itemize}

\medskip
\noindent \textbf{Rules.}
\begin{itemize}
  \item Work inside an \textbf{R Project}.
  \item Use the \textbf{Terminal tab} for Git commands (not the R Console).
  \item If something breaks: read the error, check your paths, ask a neighbor, then ask for help.
\end{itemize}

\section*{Questions}

\begin{enumerate}

  \item \textbf{Install / verify your toolchain (proof required).}
  \begin{enumerate}
    \item Confirm you can open \textbf{RStudio} and run R code in the \textbf{Console}.
    \item In the Console, run:
\begin{verbatim}
R.version.string
\end{verbatim}
    \item \textbf{Proof:} Create a file \texttt{outputs/r\_check.txt} that contains your R version string.

    \medskip
    \noindent \textit{Hint:} Use \texttt{writeLines(...)} to write text to a file.
  \end{enumerate}

  \item \textbf{Create an R Project (proof required).}
  \begin{enumerate}
    \item Create an R Project folder for this course on your computer.
    \item Open the \texttt{.Rproj} file so RStudio is working inside that project.
    \item In the Console, run:
\begin{verbatim}
getwd()
\end{verbatim}
    \item \textbf{Proof:} In \texttt{outputs/setup\_check.md}, paste the output of \texttt{getwd()} and explain (1 sentence) what it means.
  \end{enumerate}

  \item \textbf{Create a reproducible folder structure.}
  \begin{enumerate}
    \item Inside your R Project, ensure these folders exist:
    \begin{itemize}
      \item \texttt{data/}
      \item \texttt{scripts/}
      \item \texttt{outputs/}
      \item \texttt{figures/}
    \end{itemize}
    \item Write R code (in the Console or a script) that creates these folders without errors.
    \item \textbf{Proof:} In \texttt{outputs/setup\_check.md}, include the output of:
\begin{verbatim}
list.files()
\end{verbatim}
    showing the four folders.
  \end{enumerate}

  \item \textbf{Set up Git and confirm it works (proof required).}
  \begin{enumerate}
    \item Open the \textbf{Terminal tab} in RStudio.
    \item Run:
\begin{verbatim}
git --version
\end{verbatim}
    \item Configure your name and email (use the same email as your GitHub account):
\begin{verbatim}
git config --global user.name "YOUR NAME"
git config --global user.email "YOUR EMAIL"
\end{verbatim}
    \item \textbf{Proof:} In \texttt{outputs/setup\_check.md}, paste the output of:
\begin{verbatim}
git config --global user.name
git config --global user.email
\end{verbatim}
  \end{enumerate}

  \item \textbf{Clone the course repository and run a test script.}
  \begin{enumerate}
    \item In the Terminal tab, navigate to where you want the repo to live (or do this in File Explorer/Finder).
    \item Clone the course repo:
\begin{verbatim}
git clone <COURSE_REPO_URL>
\end{verbatim}
    \item Open the cloned folder as an R Project (if an \texttt{.Rproj} exists) or create one.
    \item Run a simple test in the Console (example):
\begin{verbatim}
1 + 1
\end{verbatim}
    \item \textbf{Proof:} Take a screenshot showing the Terminal tab in RStudio and save it as \texttt{outputs/terminal\_proof.png}.
  \end{enumerate}

  \item \textbf{Commit and push your first submission (proof required).}
  \begin{enumerate}
    \item Create your write-up file: \texttt{outputs/setup\_check.md}. Include:
    \begin{itemize}
      \item Your \texttt{getwd()} output
      \item Your Git name/email confirmation outputs
      \item A 2--3 sentence reflection: what broke (if anything) and what step was least intuitive
    \end{itemize}
    \item In the Terminal tab, run the Git workflow:
\begin{verbatim}
git status
git add .
git commit -m "Setup: environment + project + git"
git push
\end{verbatim}
    \item \textbf{Proof:} In \texttt{outputs/setup\_check.md}, paste:
    \begin{itemize}
      \item the output of \texttt{git status} after your commit (clean working tree), and
      \item either (a) a screenshot of GitHub showing the commit, or (b) the commit hash from:
\begin{verbatim}
git log -1
\end{verbatim}
    \end{itemize}
  \end{enumerate}

\end{enumerate}

\section*{Checklist (before you leave)}
\begin{itemize}
  \item RStudio opens and runs R code
  \item You are working inside an R Project
  \item \texttt{outputs/} and \texttt{figures/} exist
  \item Git works in the Terminal tab
  \item You cloned the course repo
  \item Your write-up and screenshot are committed and pushed to GitHub
\end{itemize}

\end{document}
