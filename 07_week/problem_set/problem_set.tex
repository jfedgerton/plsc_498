\documentclass[11pt]{article}
\usepackage{amsmath, amssymb, graphicx, hyperref}
\usepackage{enumitem}
\setlist{nosep}
\usepackage[margin=1in]{geometry}

\title{In-Class Problem Set: Distributions, Q--Q Plots, and Faceting (R + GitHub)}
\author{}
\date{}

\begin{document}
\maketitle

\noindent \textbf{Goal.} Use real conflict data to practice diagnosing distributional shape using histograms, density plots, and Q--Q plots, including comparisons across groups using faceting. You will pull the data from GitHub, build a reproducible workflow, generate required plots, interpret what they show, and submit via GitHub.

\medskip
\noindent \textbf{Dataset.} \texttt{battle\_deaths} (1141 rows; 6 variables):
\begin{itemize}
  \item \texttt{iso2c} (country code), \texttt{country}, \texttt{year}
  \item \texttt{battle\_deaths} (numeric; battle-related deaths)
  \item \texttt{region} (categorical), \texttt{income} (categorical)
\end{itemize}

\medskip
\noindent \textbf{What to submit (in your GitHub repo).}
\begin{itemize}
  \item A script file: \texttt{scripts/lab.R}
  \item A short write-up: \texttt{outputs/writeup.md}
  \item Saved figures in \texttt{figures/} (see requirements below)
\end{itemize}

\medskip
\noindent \textbf{Rules.}
\begin{itemize}
  \item Work inside an \textbf{R Project}.
  \item Use a \textbf{sequential, hard-coded workflow} (no user-defined functions).
  \item Save figures using \texttt{ggsave()} (no screenshots).
  \item Git commands must be run in the \textbf{Terminal tab}, not the R Console.
  \item Unless your lecture explicitly did otherwise, treat missing values defensibly (state what you did).
\end{itemize}

\section*{Questions}

\begin{enumerate}

  \item \textbf{Pull the data and set up your workflow (proof required).}
  \begin{enumerate}
    \item In the \textbf{Terminal tab}, run:
\begin{verbatim}
git status
git pull
\end{verbatim}
    \item Confirm the dataset file exists in your repo (path posted in the course repository).
    \item Create the standard folder structure (if missing): \texttt{scripts/}, \texttt{outputs/}, \texttt{figures/}.
    \item \textbf{Proof (write-up):} In \texttt{outputs/writeup.md}, paste:
    \begin{itemize}
      \item the output of \texttt{getwd()},
      \item the output of \texttt{list.files()} from the project root, and
      \item the output of \texttt{list.files("data")} showing the dataset file.
    \end{itemize}
  \end{enumerate}

  \item \textbf{Load and summarize \texttt{battle\_deaths}.}
  \begin{enumerate}
    \item Load the dataset into an object named \texttt{df}.
    \item Verify the six key columns exist: \texttt{country}, \texttt{year}, \texttt{battle\_deaths}, \texttt{region}, \texttt{income}.
    \item Summarize the distribution of \texttt{battle\_deaths} (min/median/mean/max is sufficient).
    \item \textbf{Proof (write-up):} Report:
    \begin{itemize}
      \item number of rows and columns,
      \item the number of unique countries,
      \item the range of years,
      \item a short summary of \texttt{battle\_deaths}.
    \end{itemize}
  \end{enumerate}

  \item \textbf{Histogram of battle deaths (baseline).}
  
  Create a histogram of \texttt{battle\_deaths}.
  \begin{itemize}
    \item You must make a clear binning choice and state it (binwidth or number of bins).
    \item If you restrict the x-axis (e.g., to reduce the influence of extreme values), you must state the rule you used.
  \end{itemize}

  \noindent Save as:
  \[
    \texttt{figures/battle\_deaths\_hist.png}
  \]

  \item \textbf{Density plot of battle deaths (baseline).}

  Create a density plot of \texttt{battle\_deaths}.
  \begin{itemize}
    \item Use the same x-axis limits as your histogram (so the two are comparable).
    \item Label the axes clearly.
  \end{itemize}

  \noindent Save as:
  \[
    \texttt{figures/battle\_deaths\_density.png}
  \]

  \item \textbf{Q--Q plot (normality check).}

  Create a Q--Q plot comparing \texttt{battle\_deaths} to a theoretical normal distribution.
  \begin{itemize}
    \item Include a Q--Q reference line.
    \item In your write-up, describe what kind of deviation you see (e.g., heavy right tail, skew).
  \end{itemize}

  \noindent Save as:
  \[
    \texttt{figures/battle\_deaths\_qq.png}
  \]

  \item \textbf{Faceting by income and region (required).}

  Create two sets of faceted distribution plots:

  \begin{enumerate}
    \item A faceted plot by \textbf{income}
    \item A faceted plot by \textbf{region}
  \end{enumerate}

  \noindent For each set, choose \textbf{one} distribution geometry that was covered in lecture (histogram or density) and facet it. Your goal is to compare how distributional shape differs across groups.

  \medskip
  \noindent \textbf{Required:}
  \begin{itemize}
    \item Use consistent axis limits across facets (so comparisons are meaningful).
    \item If some facets are too sparse to interpret, state what you did (e.g., dropped very small groups, or noted limitations).
  \end{itemize}

  \noindent Save as:
  \begin{itemize}
    \item \texttt{figures/facet\_income.png}
    \item \texttt{figures/facet\_region.png}
  \end{itemize}

  \item \textbf{Interpretation (write-up required).}

  In \texttt{outputs/writeup.md}, write 12--16 sentences addressing:
  \begin{itemize}
    \item What do the histogram and density plot suggest about skew and tail behavior?
    \item What does the Q--Q plot reveal (and why is it useful here)?
    \item Compare distributions by income: what differences (if any) stand out?
    \item Compare distributions by region: what differences (if any) stand out?
    \item Name one concrete plotting choice you made (bins, limits, faceting) and why it helped interpretability.
  \end{itemize}

  \item \textbf{Push your work to GitHub (proof required).}
  \begin{enumerate}
    \item In the \textbf{Terminal tab}, run:
\begin{verbatim}
git status
git add .
git commit -m "Distributions lab: battle_deaths faceting + QQ"
git push
\end{verbatim}
    \item \textbf{Proof (write-up):} Paste:
    \begin{itemize}
      \item the output of \texttt{git status} after committing (clean working tree), and
      \item the output of \texttt{git log -1}.
    \end{itemize}
  \end{enumerate}

\end{enumerate}

\section*{Optional challenge (if you finish early): ggridges}
Create a ridgeline density plot using \texttt{ggridges} for \texttt{battle\_deaths} grouped by \textbf{income} \emph{or} \textbf{region}.
\begin{itemize}
  \item Save as \texttt{figures/ridgeline.png}.
  \item In 4--6 sentences, explain what the ridgeline plot makes easier (or harder) to compare relative to faceting.
\end{itemize}

\section*{Checklist (before you leave)}
\begin{itemize}
  \item \texttt{scripts/lab.R} runs top-to-bottom
  \item Required figures exist in \texttt{figures/}:
  \begin{itemize}
    \item \texttt{battle\_deaths\_hist.png}, \texttt{battle\_deaths\_density.png}, \texttt{battle\_deaths\_qq.png}
    \item \texttt{facet\_income.png}, \texttt{facet\_region.png}
  \end{itemize}
  \item \texttt{outputs/writeup.md} includes interpretation + proofs
  \item Work is committed and pushed to GitHub
\end{itemize}

\end{document}
