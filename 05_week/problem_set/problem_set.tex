\documentclass[11pt]{article}
\usepackage{amsmath, amssymb, graphicx, hyperref}
\usepackage{enumitem}
\setlist{nosep}
\usepackage[margin=1in]{geometry}

\title{In-Class Problem Set: Exploring Movie Data with Distribution and Color (R + GitHub)}
\author{}
\date{}

\begin{document}
\maketitle

\noindent \textbf{Goal.} Use real movie data to practice visualizing distributions, comparing groups, and encoding multiple variables in a single plot. You will pull the dataset from GitHub, build a reproducible workflow, generate several plots, interpret what they show, and submit your work via GitHub.

\medskip
\noindent \textbf{What to submit (in your GitHub repo).}
\begin{itemize}
  \item A script file: \texttt{scripts/lab.R}
  \item A short write-up: \texttt{outputs/writeup.md}
  \item Saved figures in \texttt{figures/} (see requirements below)
\end{itemize}

\medskip
\noindent \textbf{Rules.}
\begin{itemize}
  \item Work inside an \textbf{R Project}.
  \item Use a \textbf{sequential, hard-coded workflow} (no user-defined functions).
  \item Save figures using code (\texttt{ggsave}); do not use screenshots.
  \item Git commands must be run in the \textbf{Terminal tab}, not the R Console.
\end{itemize}

\section*{Mini codebook (use this; do not guess)}

Each row represents one movie. Relevant variables include:
\begin{itemize}
  \item \texttt{budget}: Production budget in USD (0 if unavailable).
  \item \texttt{revenue}: Worldwide box office revenue in USD (0 if unavailable).
  \item \texttt{director}: Director of the movie.
  \item \texttt{runtime}: Movie length in minutes.
  \item \texttt{vote\_average}: Average user rating (0--10).
  \item \texttt{vote\_count}: Number of user votes.
  \item \texttt{popularity}: Popularity score based on user engagement.
\end{itemize}

\section*{Questions}

\begin{enumerate}

  \item \textbf{Pull the movie dataset from GitHub (proof required).}
  \begin{enumerate}
    \item In the \textbf{Terminal tab}, run:
\begin{verbatim}
git status
git pull
\end{verbatim}
    \item Confirm the movie dataset exists in your repo (location specified in the course GitHub).
    \item \textbf{Proof (write-up):} In \texttt{outputs/writeup.md}, paste:
    \begin{itemize}
      \item the output of \texttt{getwd()} from inside your R Project, and
      \item the output of \texttt{list.files("data")} showing the movie file.
    \end{itemize}
  \end{enumerate}

  \item \textbf{Load and summarize the dataset.}
  \begin{enumerate}
    \item Load the movie dataset into an object named \texttt{df}.
    \item Summarize the dataset to understand its structure.
    
    \textbf{Suggested edit:} Use \texttt{dim(df)}, \texttt{names(df)}, and a focused summary of \texttt{budget} and \texttt{revenue}.
    \item \textbf{Proof (write-up):} Report:
    \begin{itemize}
      \item number of rows and columns,
      \item the range of \texttt{budget},
      \item the range of \texttt{revenue}.
    \end{itemize}
  \end{enumerate}

  \item \textbf{Plot distributions: movie budget and revenue.}
  
  Create two histograms:
  \begin{itemize}
    \item one for \texttt{budget},
    \item one for \texttt{revenue}.
  \end{itemize}

  \noindent \textbf{Suggested edit (important):}
  \begin{itemize}
    \item Use consistent bin widths.
    \item Decide whether to include or exclude zero values, and state your choice.
  \end{itemize}

  \noindent Save the plots as:
  \begin{itemize}
    \item \texttt{figures/budget\_hist.png}
    \item \texttt{figures/revenue\_hist.png}
  \end{itemize}

  \item \textbf{Identify top-grossing directors and compare revenue.}
  \begin{enumerate}
    \item Identify the \textbf{top three directors} by total box office revenue (sum of \texttt{revenue}).
    \item Subset the data to movies directed by these three directors.
    \item Create a \textbf{boxplot} showing the distribution of \texttt{revenue} for each director.
  \end{enumerate}

  \noindent Save the plot as:
  \[
    \texttt{figures/revenue\_by\_director.png}
  \]

  \noindent \textbf{Suggested edit:} Ensure the director names are readable and the y-axis is clearly labeled in USD.

  \item \textbf{Scatter plot with size and category encodings.}
  
  Create a scatter plot with:
  \begin{itemize}
    \item x-axis: \texttt{budget}
    \item y-axis: \texttt{revenue}
  \end{itemize}

  Then:
  \begin{itemize}
    \item Choose \textbf{one additional quantitative variable} (e.g., \texttt{popularity} or \texttt{vote\_count}) and map it to \textbf{point size}.
    \item Choose \textbf{one categorical variable} (e.g., \texttt{original\_language} or a simplified genre indicator) and map it to \textbf{color}.
  \end{itemize}

  \noindent \textbf{Suggested edit:}
  \begin{itemize}
    \item Use a color palette that makes category differences clear.
    \item Avoid using size in a way that hides smaller-budget films.
  \end{itemize}

  \noindent Save the plot as:
  \[
    \texttt{figures/budget\_revenue\_scatter.png}
  \]

  \item \textbf{Interpretation (write-up required).}

  In \texttt{outputs/writeup.md}, write 10--14 sentences addressing:
  \begin{itemize}
    \item What do the budget and revenue histograms reveal about the movie industry?
    \item How do revenues differ across the top-grossing directors?
    \item In the scatter plot, what relationships are most visually salient?
    \item How do size and color encodings change what is easy or hard to see?
  \end{itemize}

  \item \textbf{Push your work to GitHub (proof required).}
  \begin{enumerate}
    \item In the \textbf{Terminal tab}, run:
\begin{verbatim}
git status
git add .
git commit -m "Movie data visualization lab"
git push
\end{verbatim}
    \item \textbf{Proof (write-up):} Paste:
    \begin{itemize}
      \item the output of \texttt{git status} after committing (clean working tree), and
      \item the output of \texttt{git log -1}.
    \end{itemize}
  \end{enumerate}

\end{enumerate}

\section*{Optional challenge (if you finish early)}
Choose one plot and create an alternative version optimized for a \textbf{general public} audience rather than an expert audience. In 5--7 sentences, explain:
\begin{itemize}
  \item what design choices you changed,
  \item what information you simplified or emphasized,
  \item and why these changes are appropriate for a public-facing visualization.
\end{itemize}

\section*{Checklist (before you leave)}
\begin{itemize}
  \item \texttt{scripts/lab.R} runs top-to-bottom
  \item \texttt{outputs/writeup.md} exists and includes interpretations + proofs
  \item Required figures exist in \texttt{figures/}
  \item Work is committed and pushed to GitHub
\end{itemize}

\end{document}
