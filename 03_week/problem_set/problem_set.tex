\documentclass[11pt]{article}
\usepackage{amsmath, amssymb, graphicx, hyperref}
\usepackage{enumitem}
\setlist{nosep}
\usepackage[margin=1in]{geometry}

\title{In-Class Problem Set: Scaling Plots with Overdispersed Election Data (R + GitHub)}
\author{}
\date{}

\begin{document}
\maketitle

\noindent \textbf{Goal.} Use overdispersed election data to practice how axis scaling changes what patterns are visible. You will (i) pull data from GitHub, (ii) build a reproducible workflow, (iii) make the same plot twice (raw vs scaled), (iv) write an interpretation comparing the two, and (v) submit via GitHub.

\medskip
\noindent \textbf{What to submit (in your GitHub repo).}
\begin{itemize}
  \item A script file: \texttt{scripts/lab.R}
  \item A short write-up: \texttt{outputs/writeup.md}
  \item Two saved figures: \texttt{figures/plot\_raw.png} and \texttt{figures/plot\_scaled.png}
\end{itemize}

\medskip
\noindent \textbf{Rules.}
\begin{itemize}
  \item Work inside an \textbf{R Project}.
  \item Use a \textbf{sequential, hard-coded workflow} (no user-defined functions).
  \item Save outputs using code (\texttt{ggsave}); do not rely on screenshots.
  \item Git commands go in the \textbf{Terminal tab} (not the R Console).
\end{itemize}

\section*{Questions}

\begin{enumerate}

  \item \textbf{Copy the data from GitHub (proof required).}
  \begin{enumerate}
    \item Pull the latest version of the course repository to ensure you have the election dataset.
    \item Confirm the dataset file exists.
    \item \textbf{Proof (write-up):} In \texttt{outputs/writeup.md}, paste:
    \begin{itemize}
      \item the output of \texttt{getwd()} (from inside your R Project), and
      \item the output of \texttt{list.files("data")} showing the dataset file.
    \end{itemize}
  \end{enumerate}

  \item \textbf{Set up a reproducible workflow (folders + script).}
  \begin{enumerate}
    \item Ensure your project contains these folders (create them if missing):
    \begin{itemize}
      \item \texttt{scripts/}
      \item \texttt{outputs/}
      \item \texttt{figures/}
    \end{itemize}
    \item Create a script named \texttt{scripts/lab.R}. All code for this problem set must live in this script.
    \item \textbf{Suggested edit (important):} At the top of \texttt{scripts/lab.R}, include:
    \begin{itemize}
      \item a short header comment describing what the script does,
      \item \texttt{library(...)} calls,
      \item \texttt{set.seed(123)}.
    \end{itemize}
    \item \textbf{Proof (write-up):} paste the output of \texttt{list.files()} from your project root.
  \end{enumerate}

  \item \textbf{Load the election data and build the analysis dataset.}
  \begin{enumerate}
    \item Load \texttt{data/HOUSE\_precinct\_general.csv} into an object called \texttt{df}.
    \item Filter the data so it includes only:
    \begin{itemize}
      \item general election entries (stage = \texttt{"GEN"})
      \item major parties only (party\_simplified in \texttt{\{"DEMOCRAT","REPUBLICAN"\}})
      \item non-missing county information
    \end{itemize}
    \item Aggregate to the \textbf{county level} and compute:
    \begin{itemize}
      \item \texttt{county\_total\_votes = DEMOCRAT + REPUBLICAN}
      \item \texttt{rep\_share = REPUBLICAN / (DEMOCRAT + REPUBLICAN)}
    \end{itemize}
    \item \textbf{Suggested edit:} Use the codebook (in the repo) to confirm the meaning of \texttt{votes}, \texttt{party\_simplified}, and \texttt{county\_name}. Cite the codebook filename in your write-up.
    \item \textbf{Proof (write-up):} report:
    \begin{itemize}
      \item number of counties in your aggregated dataset,
      \item summary of \texttt{county\_total\_votes},
      \item summary of \texttt{rep\_share}.
    \end{itemize}
  \end{enumerate}

  \item \textbf{Plot 1: raw scale (required).}
  
  Create a scatter plot with:
  \begin{itemize}
    \item x-axis: \texttt{county\_total\_votes}
    \item y-axis: \texttt{rep\_share}
    \item point color: \texttt{rep\_share} (continuous color scale; use this to reflect partisanship)
  \end{itemize}

  \noindent Save the figure as:
  \[
    \texttt{figures/plot\_raw.png}
  \]

  \noindent \textbf{Suggested edit:} Label axes clearly (what is being measured), and include a legend title.

  \item \textbf{Plot 2: scaled version (required).}
  
  Make the \emph{same} plot again, but change the scale of the x-axis to address overdispersion. Use one of:
  \begin{itemize}
    \item log scaling (e.g., log10 x-axis), or
    \item another defensible scaling choice discussed in lecture.
  \end{itemize}

  \noindent Save the figure as:
  \[
    \texttt{figures/plot\_scaled.png}
  \]

  \noindent \textbf{Suggested edit:} Make the axis label explicitly indicate the scaling choice (e.g., “log scale”).

  \item \textbf{Interpretation + GitHub submission (proof required).}
  \begin{enumerate}
    \item In \texttt{outputs/writeup.md}, write 8--12 sentences answering:
    \begin{itemize}
      \item What is mapped to x, y, and color in both plots?
      \item What is hard to see on the raw scale but easier to see on the scaled plot?
      \item What (if anything) becomes harder to interpret after scaling?
      \item If you had to show only one version to a general audience, which would you choose and why?
    \end{itemize}

    \item Commit and push your work to GitHub.

    \item \textbf{Proof (write-up):} paste:
    \begin{itemize}
      \item the output of \texttt{git status} after your commit (clean working tree), and
      \item the output of \texttt{git log -1} (one line is fine).
    \end{itemize}
  \end{enumerate}

\end{enumerate}

\section*{Optional challenge (one extra)}
Create a second scaled plot where you change the scale choice (e.g., compare log10 vs another scaling approach). In 3--5 sentences, explain which scaling choice better supports a clear comparison and why.

\section*{Checklist (before you leave)}
\begin{itemize}
  \item \texttt{scripts/lab.R} exists and runs top-to-bottom inside an R Project
  \item \texttt{figures/plot\_raw.png} exists
  \item \texttt{figures/plot\_scaled.png} exists
  \item \texttt{outputs/writeup.md} includes required interpretation and proofs
  \item Work is committed and pushed to GitHub
\end{itemize}

\end{document}
