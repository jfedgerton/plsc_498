\documentclass[11pt]{article}
\usepackage{amsmath, amssymb, graphicx, hyperref}
\usepackage{enumitem}
\setlist{nosep}
\usepackage[margin=1in]{geometry}

\title{In-Class Problem Set: Reproducible Visualization Workflow (R + GitHub)}
\author{}
\date{}

\begin{document}
\maketitle

\noindent \textbf{Goal.} Extend the code from the lecture slides to produce a small, reproducible workflow: load the provided dataset, choose variables using the codebook, create multiple figures using explicit mappings, and submit your work through GitHub.

\medskip
\noindent \textbf{What to submit (in your GitHub repo).}
\begin{itemize}
  \item A script file: \texttt{scripts/lab.R}
  \item A short write-up: \texttt{outputs/writeup.md}
  \item Saved figures: at least 4 image files in \texttt{figures/}
  \item (Optional but recommended) a log file: \texttt{outputs/log.txt}
\end{itemize}

\medskip
\noindent \textbf{Rules.}
\begin{itemize}
  \item Work inside an \textbf{R Project}.
  \item Use a \textbf{sequential, hard-coded workflow} (no user-defined functions).
  \item You may consult notes and documentation. If you use any external code, cite it in your write-up.
\end{itemize}

\section*{Questions}

\begin{enumerate}

  \item \textbf{Create an R Project (proof required).}
  \begin{enumerate}
    \item Create an R Project for this course on your computer.
    \item \textbf{Proof:} In your \texttt{outputs/writeup.md}, include:
    \begin{itemize}
      \item the output of \texttt{getwd()} run from inside the project, and
      \item a screenshot showing the \texttt{.Rproj} file in your project folder \emph{or} the RStudio Project name visible in the RStudio window.
    \end{itemize}
  \end{enumerate}

  \item \textbf{Load the provided dataset from the \texttt{data/} folder.}
  \begin{enumerate}
    \item Confirm the dataset file exists in \texttt{data/}. (Do not manually move it.)
    \item Write code in \texttt{scripts/lab.R} to load it into R as an object named \texttt{df}.
    \item \textbf{Proof:} In \texttt{outputs/writeup.md}, include:
    \begin{itemize}
      \item the dimensions of \texttt{df} (rows $\times$ columns), and
      \item the first 3 column names.
    \end{itemize}
  \end{enumerate}

  \item \textbf{Select variables using the codebook.}
  \begin{enumerate}
    \item Consult the dataset codebook and choose:
    \begin{itemize}
      \item \textbf{two continuous} variables (numeric, meaningfully ordered with many values), and
      \item \textbf{two categorical} variables (groups/labels).
    \end{itemize}
    \item In \texttt{outputs/writeup.md}, list the four variables and briefly justify (1 sentence each) why they are continuous vs categorical.
    \item \textbf{Proof:} Include either \texttt{str(df)} output for the four variables \emph{or} a small table showing each variable and its class/type.
  \end{enumerate}

  \item \textbf{Create a reproducible folder structure + (optional) logging.}
  \begin{enumerate}
    \item Ensure these folders exist in your project:
    \begin{itemize}
      \item \texttt{scripts/}
      \item \texttt{outputs/}
      \item \texttt{figures/}
      \item \texttt{logs/} \textit{(optional but recommended)}
    \end{itemize}
    \item In \texttt{scripts/lab.R}, add code that creates any missing directories (without errors).
    \item \textbf{Proof:} In \texttt{outputs/writeup.md}, include \texttt{list.files()} output showing the folders.
    \item \textbf{Optional (challenge):} Create a simple log file \texttt{outputs/log.txt} that records:
    \begin{itemize}
      \item the current date/time,
      \item the dataset filename loaded,
      \item and the names of the four selected variables.
    \end{itemize}
  \end{enumerate}

  \item \textbf{Make three plot extensions + comment on them.}
  
  Using the lecture code as your baseline, create \textbf{three} distinct extensions (three separate figures). Each figure must include a caption in your write-up that explains:
  \begin{itemize}
    \item what variables are mapped to what visual properties,
    \item what comparison is easiest to make,
    \item and one default choice you are accepting (or changing) and why.
  \end{itemize}

  \medskip
  \noindent Your three extensions must come from different categories below (choose any three):
  \begin{enumerate}
    \item \textbf{Add a mapping:} map a categorical variable to \texttt{color} or \texttt{shape}.
    \item \textbf{Change the mark:} switch to a different geometry appropriate for the variable types (e.g., boxplot for outcome vs group).
    \item \textbf{Add an annotation layer (lightweight):} add a title + axis labels + a one-sentence caption in the write-up.
    \item \textbf{Handle overplotting:} use transparency (\texttt{alpha}) and briefly explain why.
    \item \textbf{Re-order categories:} reorder a categorical axis to improve interpretability (explain the ordering rule).
  \end{enumerate}

  \medskip
  \noindent \textbf{Saving requirement:} Save each plot to \texttt{figures/} using \texttt{ggsave()} (do not rely on screenshots). Name files clearly (e.g., \texttt{figures/plot1.png}, \texttt{figures/plot2.png}, \texttt{figures/plot3.png}).

  \item \textbf{Public vs expert visualization + GitHub submission (proof required).}
  \begin{enumerate}
    \item Create \textbf{two} versions of the \emph{same} visualization:
    \begin{itemize}
      \item one intended for a \textbf{general public} audience, and
      \item one intended for an \textbf{expert} audience.
    \end{itemize}
    \item In \texttt{outputs/writeup.md}, state at least \textbf{three decision rules} you used to adapt the design (e.g., labeling density, uncertainty/context, annotation, choice of scale, what to simplify vs keep).
    \item \textbf{Challenge ideas (pick one):}
    \begin{itemize}
      \item Add a short ``limitations'' note for the public version (1--2 sentences).
      \item Add a technical note for the expert version describing a key default or transformation.
      \item Add a small multiple (two panels) for the expert version only.
      \item Add a deliberately ``bad'' version and write 3 bullets on why it misleads.
    \end{itemize}
    \item \textbf{GitHub requirement:} Commit and push your work.
    
    \textbf{Proof:} In \texttt{outputs/writeup.md}, include:
    \begin{itemize}
      \item the output of \texttt{git status} \emph{after} committing (showing a clean working tree), and
      \item either a screenshot of your GitHub repo showing the latest commit \emph{or} the commit hash and message.
    \end{itemize}
  \end{enumerate}

\end{enumerate}

\section*{Checklist (before you leave)}
\begin{itemize}
  \item \texttt{scripts/lab.R} exists and runs top-to-bottom
  \item \texttt{outputs/writeup.md} exists and includes required proofs
  \item At least 4 figures saved in \texttt{figures/} (3 extensions + 2 audience versions can overlap if you clearly label)
  \item Work is pushed to GitHub
\end{itemize}

\end{document}
