\documentclass[11pt]{article}
\usepackage{amsmath, amssymb, graphicx, hyperref}
\usepackage{enumitem}
\setlist{nosep}
\usepackage[margin=1in]{geometry}

\title{In-Class Problem Set: Totals vs Proportions in Political and Health Data (R + GitHub)}
\author{}
\date{}

\begin{document}
\maketitle

\noindent \textbf{Goal.} Practice the difference between visualizing \emph{totals} and \emph{proportions}, and understand how these choices change interpretation. You will use state-level election and public health data to construct total counts, proportions, and grouped comparisons, then visualize them using appropriate color encodings.

\medskip
\noindent \textbf{Dataset.} The provided dataset contains one row per U.S.\ state with election results and cumulative death counts through November 7, 2020.

\medskip
\noindent \textbf{Key variables (excerpt).}
\begin{itemize}
  \item \texttt{totalvotes}, \texttt{biden\_votes}, \texttt{trump\_votes}
  \item \texttt{biden\_share}, \texttt{trump\_share}, \texttt{biden\_margin}
  \item \texttt{covid\_deaths\_to\_2020\_11\_07}
  \item \texttt{pneumonia\_deaths\_to\_2020\_11\_07}
\end{itemize}

\medskip
\noindent \textbf{What to submit (in your GitHub repo).}
\begin{itemize}
  \item A script file: \texttt{scripts/lab.R}
  \item A short write-up: \texttt{outputs/writeup.md}
  \item Saved figures in \texttt{figures/} (see requirements below)
\end{itemize}

\medskip
\noindent \textbf{Rules.}
\begin{itemize}
  \item Work inside an \textbf{R Project}.
  \item Use a \textbf{sequential, hard-coded workflow} (no user-defined functions).
  \item Save figures using \texttt{ggsave()} (no screenshots).
  \item Git commands must be run in the \textbf{Terminal tab}, not the R Console.
  \item Color choices must be intuitive and accessibility-conscious.
\end{itemize}

\section*{Questions}

\begin{enumerate}

  \item \textbf{Pull the data and set up your workflow (proof required).}
  \begin{enumerate}
    \item In the \textbf{Terminal tab}, run:
\begin{verbatim}
git status
git pull
\end{verbatim}
    \item Confirm the dataset exists in your repo (path specified in the course GitHub).
    \item Create the standard folder structure if missing: \texttt{scripts/}, \texttt{outputs/}, \texttt{figures/}.
    \item \textbf{Proof (write-up):} In \texttt{outputs/writeup.md}, paste:
    \begin{itemize}
      \item the output of \texttt{getwd()},
      \item the output of \texttt{list.files("data")}.
    \end{itemize}
  \end{enumerate}

  \item \textbf{Load and inspect the dataset.}
  \begin{enumerate}
    \item Load the dataset into an object named \texttt{df}.
    \item Summarize the data structure.
    
    \textbf{Suggested edit:} Use \texttt{dim(df)}, \texttt{names(df)}, and a focused summary of vote and death variables.
    \item \textbf{Proof (write-up):} Report:
    \begin{itemize}
      \item number of rows and columns,
      \item the range of \texttt{totalvotes},
      \item the range of \texttt{covid\_deaths\_to\_2020\_11\_07}.
    \end{itemize}
  \end{enumerate}

  \item \textbf{Create a total illness death variable.}
  \begin{enumerate}
    \item Create a new variable:
    \[
      \texttt{total\_illness\_deaths = covid\_deaths + pneumonia\_deaths}
    \]
    \item Compute total illness deaths aggregated across all states.
    \item \textbf{Proof (write-up):} Report the national total and the minimum/maximum state totals.
  \end{enumerate}

  \item \textbf{Visualize totals: bar plot of total illness deaths by state.}
  
  Create a bar plot showing \textbf{total illness deaths by state}.

  \medskip
  \noindent \textbf{Required:}
  \begin{itemize}
    \item Order states by total illness deaths.
    \item Clearly label axes.
    \item State explicitly that this is a \emph{total}, not a proportion.
  \end{itemize}

  \noindent Save as:
  \[
    \texttt{figures/total\_illness\_deaths\_by\_state.png}
  \]

  \item \textbf{Visualize proportions: Trump vs Biden.}

  Create a categorical variable indicating the election winner:
  \begin{itemize}
    \item \texttt{winner = "Biden"} if \texttt{biden\_share > trump\_share}
    \item \texttt{winner = "Trump"} otherwise
  \end{itemize}

  Then:
  \begin{enumerate}
    \item Create a bar plot showing the \textbf{proportion of total illness deaths} accounted for by Trump- vs Biden-won states.
    \item Use an appropriate color scheme:
    \begin{itemize}
      \item red--blue for categorical winner, or
      \item red--purple--blue if you choose to encode \texttt{biden\_margin}.
    \end{itemize}
  \end{enumerate}

  \noindent \textbf{Required:}
  \begin{itemize}
    \item Colors must be intuitive and colorblind-friendly.
    \item The plot must clearly communicate that values are \emph{proportions}, not raw counts.
  \end{itemize}

  \noindent Save as:
  \[
    \texttt{figures/illness\_deaths\_by\_winner\_proportion.png}
  \]

  \item \textbf{Interpretation (write-up required).}

  In \texttt{outputs/writeup.md}, write 12--16 sentences addressing:
  \begin{itemize}
    \item How interpretation changes when moving from totals to proportions.
    \item What the bar plot of totals emphasizes that the proportional plot hides.
    \item What the proportional plot emphasizes that the totals plot hides.
    \item Why your color choices are appropriate for the task.
    \item One way these plots could be misinterpreted if shown without context.
  \end{itemize}

  \item \textbf{Push your work to GitHub (proof required).}
  \begin{enumerate}
    \item In the \textbf{Terminal tab}, run:
\begin{verbatim}
git status
git add .
git commit -m "Totals vs proportions lab"
git push
\end{verbatim}
    \item \textbf{Proof (write-up):} Paste:
    \begin{itemize}
      \item the output of \texttt{git status} after committing,
      \item the output of \texttt{git log -1}.
    \end{itemize}
  \end{enumerate}

\end{enumerate}

\section*{Optional challenge (if you finish early)}
Create an alternative visualization where:
\begin{itemize}
  \item the same data are shown using \textbf{faceting} instead of color, or
  \item you encode \texttt{biden\_margin} as a continuous color scale (red--purple--blue).
\end{itemize}
In 5--7 sentences, explain which version is clearer and for what audience.

\section*{Checklist (before you leave)}
\begin{itemize}
  \item \texttt{scripts/lab.R} runs top-to-bottom
  \item Required figures exist in \texttt{figures/}
  \item \texttt{outputs/writeup.md} includes interpretation + proofs
  \item Work is committed and pushed to GitHub
\end{itemize}

\end{document}
